\documentclass[12pt]{article}
\usepackage[margin=1in]{geometry}
\usepackage{hyperref}
\hypersetup{
    colorlinks = true,
    urlcolor   = blue,
    linkcolor  = black,
    citecolor  = black
}

\begin{document}

\title{Committee Assignment with Excel Data: README}
\author{}
\date{}
\maketitle

\section{Introduction}

This document provides a step-by-step guide for using the Python script that
\begin{enumerate}
    \item Reads an Excel file containing students and their committee ratings,
    \item Solves the assignment problem (maximizing total rating, respecting minimum/maximum committee size constraints),
    \item Exports the result to a new Excel file that details each committee's assigned students.
\end{enumerate}

The solution is formulated as an Integer Linear Program (ILP) using the
\texttt{PuLP} library (\url{https://github.com/coin-or/pulp}).
The default solver is CBC, provided with PuLP.
For a typical problem size (e.g., 100 students and 10 committees), the solution
often completes in well under a second on a typical laptop.

\section{Requirements}

\begin{itemize}
    \item \textbf{Python 3.7+} (or later)
    \item \textbf{pip} (Python package manager)
\end{itemize}

\section{Installation}

\begin{enumerate}
    \item Clone or download this repository (containing the Python script).
    \item Install required Python libraries:
          \begin{verbatim}
pip install pulp pandas openpyxl
    \end{verbatim}
    \item \texttt{pulp} is used for ILP solving,
          \texttt{pandas} for reading/writing Excel,
          and \texttt{openpyxl} for \texttt{.xlsx} support.
\end{enumerate}

\section{Data Format (Excel)}

Your \textbf{input Excel file} should have:
\begin{itemize}
    \item \textbf{Column A}: Student names (or IDs), one row per student.
    \item \textbf{Columns B..M}: Numerical ratings for each committee.
          If there are $m$ committees, you should have $m$ columns of ratings
          after the first name column.
\end{itemize}

An example with 3 committees:

\begin{center}
    \begin{tabular}{lrrr}
        \hline
        Name  & Comm0 & Comm1 & Comm2 \\
        \hline
        Alice & 10    & 3     & 5     \\
        Bob   & 2     & 9     & 1     \\
        Carol & 7     & 4     & 10    \\
        \hline
    \end{tabular}
\end{center}

\section{How to Use}

\subsection{Step 1: Prepare the Excel File}

\begin{enumerate}
    \item Create (or update) a file named, for example, \texttt{students\_committees.xlsx} with the described format.
    \item Make sure there are no extra blank rows or header rows that might break the parsing.
\end{enumerate}

\subsection{Step 2: Configure min/max Constraints}

Open \texttt{committee\_assignment\_excel\_final.py} (or similarly named script)
and look for the following lines in the \verb|if __name__ == "__main__":| block:

\begin{verbatim}
# Example: a single global min=2, max=10 for all committees
global_min_value = 2
global_max_value = 10

# Or define per-committee arrays...
min_sizes = None
max_sizes = None
\end{verbatim}

You can set constraints in three ways:

\begin{enumerate}
    \item \textbf{Global constraints for all committees:}\\
          Set \texttt{global\_min\_value} and \texttt{global\_max\_value} to your desired integers
          (e.g.\ 2 and 10). Then keep \texttt{min\_sizes = None} and \texttt{max\_sizes = None}.

    \item \textbf{Per-committee constraints:}\\
          Let \texttt{min\_sizes = [m1, m2, ..., m\_j]}
          and \texttt{max\_sizes = [M1, M2, ..., M\_j]}, one entry per committee, and set
          \texttt{global\_min\_value = None} and \texttt{global\_max\_value = None}.

    \item \textbf{Combine both:}\\
          If you pass both global and per-committee values, the code merges them:
          \[
              \mathrm{final\_min}[j] = \max(\mathrm{min\_sizes}[j], \mathrm{global\_min\_value}),
          \]
          \[
              \mathrm{final\_max}[j] = \min(\mathrm{max\_sizes}[j], \mathrm{global\_max\_value}).
          \]
\end{enumerate}

\subsection{Step 3: Run the Script}

After configuring the constraints, run:

\begin{verbatim}
python committee_assignment_excel_final.py
\end{verbatim}

\begin{itemize}
    \item The script reads the file (e.g. \texttt{students\_committees.xlsx})
          and sets up an ILP to maximize the sum of ratings.
    \item It solves the assignment using the default CBC solver.
    \item By default, it writes the results to \texttt{results.xlsx}.
\end{itemize}

You can change file names by editing:

\begin{verbatim}
excel_input = "students_committees.xlsx"
excel_output = "results.xlsx"
\end{verbatim}

\subsection{Step 4: Inspect the Output}

The script creates \texttt{results.xlsx} with a single sheet named \emph{"Assignments"}.
Each committee is placed in its own column. Within each column:

\begin{itemize}
    \item Row 1: \texttt{Committee j}
    \item Row 2: \texttt{Total rating: X}
    \item Row 3: \texttt{Number assigned: Y}
    \item Rows 4+: The names of the assigned students
\end{itemize}

If committees differ in number of assigned students, extra cells will remain blank.
For instance:

\begin{center}
    \begin{tabular}{lll}
        \textbf{Committee\_0} & \textbf{Committee\_1} & \textbf{Committee\_2} \\
        \hline
        Committee 0           & Committee 1           & Committee 2           \\
        Total rating: 15      & Total rating: 22      & Total rating: 18      \\
        Number assigned: 2    & Number assigned: 3    & Number assigned: 1    \\
        Alice                 & Bob                   & Carol                 \\
        David                 & Eve                   &                       \\
                              & Frank                 &                       \\
    \end{tabular}
\end{center}

\section{Customization}

\begin{itemize}
    \item \textbf{Alternative Solvers}:
          If you have Gurobi, CPLEX, or GLPK, you can replace
          \verb|prob.solve()| in the code with e.g. \verb|prob.solve(pulp.GUROBI_CMD())|.
    \item \textbf{Additional Constraints}:
          You can add new constraints (e.g., a specific student cannot join certain committees)
          by editing the \verb|solve_committee_assignment| function.
    \item \textbf{Output Format}:
          If you want each committee on a separate sheet or a different layout,
          modify the section in \verb|solve_from_excel| that exports the \verb|results.xlsx|.
\end{itemize}

\section{FAQ}

\paragraph{Q: Why might I see \texttt{"No optimal solution found!"}?}
\begin{itemize}
    \item If the sum of minimum constraints exceeds the number of students, or
          if the sum of maximum constraints is less than the number of students,
          the problem is infeasible.
    \item Alternatively, the solver might fail or be missing. Confirm that
          \texttt{pulp} and \texttt{CBC} are installed correctly.
\end{itemize}

\paragraph{Q: How fast does it run?}
\begin{itemize}
    \item For about 100 students and 10 committees, the default CBC solver
          typically finds an optimal solution in less than a second.
\end{itemize}

\paragraph{Q: Can I read from CSV instead of Excel?}
\begin{itemize}
    \item Yes, replace \verb|pd.read_excel(...)| with \verb|pd.read_csv(...)|.
          Make sure to adjust for any header rows or column differences.
\end{itemize}

\end{document}
